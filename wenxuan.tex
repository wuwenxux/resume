% !TEX TS-program = xelatex
% !TEX encoding = UTF-8 Unicode
% !Mode:: "TeX:UTF-8"

\documentclass{resume}
\usepackage{zh_CN-Adobefonts_external} % Simplified Chinese Support using external fonts (./fonts/zh_CN-Adobe/)
\usepackage{linespacing_fix} % disable extra space before next section
\usepackage{cite}

\begin{document}
\pagenumbering{gobble} % suppress displaying page number

\name{邬文轩}

\basicInfo{
  \email{wwx0306@foxmail.com} \textperiodcentered\ 
  \phone{(+86) 191-159-46187} \textperiodcentered\ 
  \linkedin[Wenxuan Wu]{https://www.linkedin.com/in/文轩-邬-5704b88b}}
\section{\faInfo\ 个人优势 }
\begin{itemize}[parsep=0.5ex]
	\item 多年服务器、嵌入式(Linux、Android)开发经验、对内核网络协议栈和应用开发非常熟悉
	\item 熟悉TCP/IP协议栈,并设计和参与开发过新的传输层协议
	\item 对网络维护和网络驱动开发有丰富经验,熟悉开源社区运营规则,并对某些知名项目有过贡献Kernel, DPDK等
	\item 对网络性能问题有较深入积累和优化,时延、吞吐等相关优化方案
\end{itemize}

\section{\faUsers\ 工作经历}
\datedsubsection{\textbf{鼎桥科技有限公司}, 成都}{2023年9月 -- }
\role{资深开发工程师} {gdb  c  git dpdk netstack}
\begin{onehalfspacing}
模组终端产品R17维护以及开发
\begin{itemize}
	\item 终端产品在不同硬件(mips, arm, x86)平台下性能问题分析定位
	\item 分析、定位、解决客户和FAE反馈的问题
\end{itemize}
\end{onehalfspacing}

\datedsubsection{\textbf{成都海网技术有限公司}, 成都}{2022年7月 -- 2023年8月}
\role{dpdk 网络协议栈开发工程师}{gdb  c  git dpdk netstack}
\begin{onehalfspacing}
海网传输层协议特性实现
\begin{itemize}
  \item 独立完成seanet传输层网络节点流量生成工具,基于海网(seanet)的专有节点,流量统计(overlay,multipath,隐匿)
  \item 独立完成基于seanet传输层协议特性开发以及维护(乱序重排、类似tcp的nack以及ack基于seanet网络的传输层实现及优化)
  \item 完成seanet已有的缓存模块资产沉淀,包括完整的说明文档以及使用说明文档撰写,bug维护,特性开发。
  \item 作为maintainner,review代码并负责合并。
\end{itemize} 
\end{onehalfspacing}

\begin{onehalfspacing}
	Android 底层通信机制实现 {C 蓝牙协议}
\begin{itemize}
  \item 蓝牙协议分包支持
  \item upnp协议分析
\end{itemize} 
\end{onehalfspacing}

\datedsubsection{\textbf{中科创达有限公司}, 成都}{2021年11月 -- 2022年7月}
\role{驱动开发工程师}{gdb  c  git dpdk netstack}
\begin{onehalfspacing}
Intel PMD driver 维护开发 
\begin{itemize}
	\item Intel PMD driver 开发
	\item 实现了 Header Split feature 开发
	\item Bug 定位、复现、方案修复(十二个bug)
	\item 对英特尔不同产品线网卡驱动(ixgbe, i40e, ice)Bug Root Cause分析、修复、方案讨论
	\item Feature开发(基于810系列网卡的RTP协议切分,可以实现视频数据直接导入显卡内存)
\end{itemize} 
\end{onehalfspacing}

\datedsubsection{\textbf{baicells 技术有限公司}, 成都}{2019年8月 -- 2021年11月}
\role{Linux 终端开发工程师}{C  shell gdb kernel 网络协议栈}
\begin{onehalfspacing}
nw户态解析程序
\begin{itemize}
	\item 配置文件读取、加载、重载、删除、显示
	\item 输入校验、异常处理、日志记录
	\item 客户反馈故障,bug分析
	\item Bug fix, 重构
\end{itemize}
nw内核驱动模块
	\begin{itemize}
	\item mptcp 内核模块功能支持
	\item 内核模块收包性能问题分析(性能、稳定性)输入参数校验
	\item 内核模块驱动收包流程分析,以及用户态参数不生效问题分析
	\end{itemize}
\end{onehalfspacing}

\datedsubsection{\textbf{国家电网}, 地级市电网}{2013 年7月 -- 2019年7月}
\role{信息网络运维工程师}{网络规划、配置、策略部署、网络异常分析}
\begin{onehalfspacing}
地级市通信网络、应用维护
\begin{itemize}
\item 电网相关信息系统维护,规划
\item 电网相关设备网络故障分析
\item 主导并参与电网相关科技项目预研、实现、落地 
\end{itemize}
\end{onehalfspacing}

\section{\faGraduationCap\  教育背景}
\datedsubsection{\textbf{四川大学}, 四川,成都}{2010 -- 2013}
\textit{硕士研究生}\ 软件工程
\datedsubsection{\textbf{四川大学}, 四川, 成都}{2006 -- 2010}
\textit{学士}\ 软件工程

\section{\faCogs\ IT 技能}
% increase linespacing [parsep=0.5ex]
\begin{itemize}[parsep=0.5ex]
  \item C Bash Linux ops Office Suite gdb vim vscode
  \item 网络协议熟悉,接触的工作框架熟悉,体系结构熟悉,常用数据结构熟悉,计算机基础熟悉
  \item Git 常用操作 
\end{itemize}

\section{\faInfo\ 其他}
% increase linespacing [parsep=0.5ex]
\begin{itemize}[parsep=0.5ex]
  \item GitHub: https://github.com/wuwenxux
  \item 英语  听说读写熟练(TOEFL 92) CET 6级
\end{itemize}

%% Reference
%\newpage
%\bibliographystyle{IEEETran}
%\bibliography{mycite}
\end{document}
