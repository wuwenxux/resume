% !TEX TS-program = xelatex
% !TEX encoding = UTF-8 Unicode
% !Mode:: "TeX:UTF-8"

\documentclass{resume}
\usepackage{zh_CN-Adobefonts_external} % Simplified Chinese Support using external fonts (./fonts/zh_CN-Adobe/)
% \usepackage{NotoSansSC_external}
% \usepackage{NotoSerifCJKsc_external}
% \usepackage{zh_CN-Adobefonts_internal} % Simplified Chinese Support using system fonts
\usepackage{linespacing_fix} % disable extra space before next section
\usepackage{cite}

\begin{document}
\pagenumbering{gobble} % suppress displaying page number

\name{邬文轩}

\basicInfo{
  \email{wwx0306@gmail.com} \textperiodcentered\ 
  \phone{(+86) 191-159-46187} \textperiodcentered\ 
  \linkedin[wuwenxuan]{https://www.linkedin.com/in/文轩-邬-5704b88b}
}
\section{\faUsers\ 工作经历}

\datedsubsection{\textbf{Baicells}成都 四川}{2020.06 -- 至今}
\role{终端研发工程师}{CPE MPTCP功能支持公司新功能预研开发}
\begin{onehalfspacing}
\textit{CPE NW Multipath TCP} Nw cli用户功能开发 https://github.com/wwx0306/nw
\begin{itemize}
  \item NW客户端cli用户命令解析。
  \item NW配置文件读取、加载、重载、删除。
  \item NW配置信息显示。
  \item 输入校验、异常处理、日志记录。
  \item kernel端用户空间IPC规划,调试。
  \item 客户反馈故障分析、一些开发相关的功能测试、文档撰写
  \item CPE 相关需求开发、客户、测试反馈,bug分析
  \item kernel nw 配置模块命令分析
\end{itemize}
\end{onehalfspacing}

\datedsubsection{\textbf{国家电网} 呼和浩特}{2013.07 -- 2019.09}
\role{信息网络工程师}{\textit{Linux 网络协议栈 wireshark Office Suite}}
通信网络维护
\begin{itemize}
  \item 所辖区域通信网络交换机、路由器、防火墙、电网内部系统(财务、ERP、GIS、工作票相关后台、变电站相关系统)维护。
  \item 通信设备检修:交换机、光端机、路由器
  \item 日常网络运维:网络规划(ip网段划分)、路由策略设置、组网设置、防火墙安全策略设定、常见的服务器性能指标监测。
  \item 系统维护:分析故障原因、性能指标、和客户上报的一些技术缺陷。
  \item 作为评标专家,参与若干内部网络项目标书审核、参与项目评估。
\end{itemize}

\section{\faGraduationCap\  教育背景}
\datedsubsection{\textbf{四川大学}, 成都}{2010 -- 2013}
\textit{硕士研究生}\ 软件工程, 2013
\datedsubsection{\textbf{四川大学}, 成都}{2006 -- 2010}
\textit{学士}\ 软件工程

% Reference Test
%\datedsubsection{\textbf{Paper Title\cite{zaharia2012resilient}}}{May. 2015}
%An xxx optimized for xxx\cite{verma2015large}
%\begin{itemize}
%  \item main contribution
%\end{itemize}

\section{\faCogs\ IT 技能}
% increase linespacing [parsep=0.5ex]
\begin{itemize}[parsep=0.5ex]
	\item 语言:\textit{C shell bash Rust C++ Python Office Suite VSCode} 
  \item 平台: Linux  Windows	  
  \item 开发技能: 网络协议栈、操作系统相关、内核用户空间通讯、嵌入式、常用数据结构实现使用
\end{itemize}

\section{\faHeartO\ 获奖情况}
\datedline{\textit{信息通信先进工作班组}		信息通信处}{2015年}
\datedline{\textit{决胜团队Google 公益}			Google益暖中华}{2012年}
\datedline{\textit{硕士研究生三等奖学金}		四川大学}{2012年}
\datedline{\textit{硕士研究生二等奖学金}		四川大学}{2010年}
\datedline{\textit{三等奖}                    		国家大学生信息安全大赛}{2010年}

\section{\faInfo\ 其他}
% increase linespacing [parsep=0.5ex]
\begin{itemize}[parsep=0.5ex]
  \item GitHub: https://github.com/wwx0306
  \item 英语: 四、六级 TOEFL 92
\end{itemize}
%% Reference
%\newpage
%\bibliographystyle{IEEETran}
%\bibliography{mycite}
\end{document}
